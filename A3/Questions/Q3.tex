\section{}
Consider the Laplace Transform of $f(t)$ given by
\[
    F(s) = \frac{16s^2 + 23s + 13}{(s + 1)^2(s + 2)}
\]

\begin{enumerate}[label=(\alph*)]
    \item What are the poles of $F(s)$?
    \item Using the Final Value Theorem, predict the behaviour of $f(t)$ as $t \to \infty$
    \item By hand, obtain the partial fraction expansion of $F(s)$
    \item Use MATLAB's \texttt{partfrac} to confirm your result; provide the input commands you used.
    \item Obtain $f(t)$, and confirm the prediction from (b) holds. Hint: $\lim_{t \to \infty} te^{-t} = 0$
\end{enumerate}

\subsection{}
The poles of $F(s)$ are $s = -1$ and $s = -2$. The pole at $s = -1$ has a multiplicity of 2.

\subsection{}
Using the Final Value Theorem, we can see the contribution of each pole.

\begin{enumerate}
    \item $s = -1$ has a multiplicity of 2, so the contribution will be $e^{-t}$ and $te^{-t}$.
    \item $s = -2$ has a multiplicity of 1, so the contribution will be $e^{-2t}$.
\end{enumerate}

Therefore, as $t \to \infty$, $f(t) \to 0$. This is statement 3 of the Final Value Theorem in the course notes.

\subsection{}
By the Heaviside Cover-up Method, we can obtain the partial fraction expansion of $F(s)$.

\[
\begin{aligned}
    F(s) &= \frac{16s^2 + 23s + 13}{(s + 1)^2(s + 2)} \\
    &= \frac{A}{s + 1} + \frac{B}{(s + 1)^2} + \frac{C}{s + 2} \\
    &= \frac{A(s + 1)(s + 2) + B(s + 2) + C(s + 1)^2}{(s + 1)^2(s + 2)} \\
    &= \frac{(A + C)s^2 + (3A + 2B + 2C)s + (2A + B + C)}{(s + 1)^2(s + 2)}
\end{aligned}
\]

By cover-up,
\[
\begin{aligned}
    B &= \frac{16s^2 + 23s + 13}{(s + 2)} \bigg|_{s = -1} \\
    &= \frac{16(-1)^2 + 23(-1) + 13}{(-1 + 2)} \\
    &= 6 \\
    C &= \frac{16s^2 + 23s + 13}{(s + 1)^2} \bigg|_{s = -2} \\
    &= \frac{16(-2)^2 + 23(-2) + 13}{(-2 + 1)^2} \\
    &= 31 \\
    A &= \frac{d}{ds} \left[ \frac{16s^2 + 23s + 13}{(s + 2)} \right] \bigg|_{s = -1} \\
    &= \frac{(s+2)(32s + 23) - (16s^2 + 23s + 13)(1)}{(s + 2)^2} \bigg|_{s = -1} \\
    &= \frac{(-1+2)(-32 + 23) - (16 - 23 + 13)}{(-1 + 2)^2} \\
    &= -15
\end{aligned}
\]

Solving this system of equations, we get $A = -15$, $B = 6$, and $C = 31$.

The partial fraction expansion of $F(s)$ is therefore:
\begin{empheq}[box=\fbox]{align*}
    F(s) = \frac{16s^2 + 23s + 13}{(s + 1)^2(s + 2)} &= \frac{-15}{s + 1} + \frac{6}{(s + 1)^2} + \frac{31}{s + 2}
\end{empheq}

\subsection{}
Using MATLAB's \texttt{partfrac} command, we get the following result.

\begin{verbatim}
>> syms s
>> partfrac(16*s^2 + 23*s + 13, (s + 1)^2*(s + 2))

ans =

    6/(s + 1)^2 - 15/(s + 1) + 31/(s + 2)
\end{verbatim}

\subsection{}
Using the inverse Laplace Transform, we get the following result.

\[
\begin{aligned}
    f(t) &= \mathcal{L}^{-1} \left\{ \frac{16s^2 + 23s + 13}{(s + 1)^2(s + 2)} \right\} \\
    &= \mathcal{L}^{-1} \left\{ \frac{-15}{s + 1} + \frac{6}{(s + 1)^2} + \frac{31}{s + 2} \right\} \\
    &= \mathcal{L}^{-1} \left\{ \frac{-15}{s + 1} \right\} + \mathcal{L}^{-1} \left\{ \frac{6}{(s + 1)^2} \right\} + \mathcal{L}^{-1} \left\{ \frac{31}{s + 2} \right\} \\
\end{aligned}
\]

From the table of Laplace Transforms, the following Laplace relations are of use:
\[
\begin{aligned}
    \mathcal{L}\{t e^{at}\} &= \frac{1}{(s - a)^2} \\
    \mathcal{L}\{e^{at}\} &= \frac{1}{s - a}
\end{aligned}    
\]

Therefore,

\[
\begin{aligned}
    \boxed{f(t) = -15e^{-t} + 6te^{-t} + 31e^{-2t}}
\end{aligned}
\]

Again this can be verified by Matlab using the following commands.
\begin{verbatim}
>> syms t
>> ilaplace(partfrac(16*s^2 + 23*s + 13, (s + 1)^2*(s + 2)))

ans =
    6/(s + 1)^2 - 15/(s + 1) + 31/(s + 2)
\end{verbatim}

Taking the limit as $t \to \infty$, we get the following result.
\begin{empheq}[box=\fbox]{align*}
    \lim_{t \to \infty} f(t) &= \lim_{t \to \infty} \left(-15e^{-t} + 6te^{-t} + 31e^{-2t}\right)\\
    &= 0 + 0 + 0 \\
    &= 0
\end{empheq}

