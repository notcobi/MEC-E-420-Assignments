\section{}
Consider the state-space form:
\[
\begin{aligned}
    \begin{bmatrix}
        \dot{x_1} \\
        \dot{x_2}
    \end{bmatrix}
    &=
    \begin{bmatrix}
        2 -1 \\
        -1 2
    \end{bmatrix}
    \begin{bmatrix}
        x_1 \\
        x_2
    \end{bmatrix}
    + 
    \begin{bmatrix}
        2 \\
        0
    \end{bmatrix}
    u \\
    y &=
    \begin{bmatrix}
        0 & 3
    \end{bmatrix}
    \begin{bmatrix}
        x_1 \\
        x_2
    \end{bmatrix}
\end{aligned}
\]

With the the initial conditions:
\[
\begin{aligned}
    \begin{bmatrix}
        x_1(0) \\
        x_2(0)
    \end{bmatrix}
    &=
    \begin{bmatrix}
        -1 \\
        1
    \end{bmatrix}\\
    u(t) &= 1_{+}(t)
\end{aligned}
\]

In Assignment \#2, you found that
\[
\begin{aligned}
        A &=
        \begin{bmatrix}
            2 & -1 \\
            -1 & 2
        \end{bmatrix} 
        \implies 
        e^{At} =
        \frac{1}{2}
        \begin{bmatrix}
            e^t + e^{3t} & e^t - e^{3t} \\
            e^t - e^{3t} & e^t + e^{3t}
        \end{bmatrix} \\
\end{aligned}
\]

\begin{enumerate}[label=(\alph*)]
    \item Find the zero-input trajectory $x_{z-i}(t)$ and response $y_{z-i}(t)$ using hand calculations
    \item Find the zero-state trajectory $x_{z-s}(t)$ and response $y_{z-s}(t)$ using hand calculations
    \item Find the trajectory $x(t)$ and response $y(t)$ of the system using hand calculations
    \item Redo (a)-(c) using only MATLAB's symbolic toolbox, and ensure your solutions match. Provide the input commands that you used.
\end{enumerate}

\subsection{}
The zero-input trajectory problem is given by:
\[
\begin{aligned}
    \dot{x} &= Ax \\
    x(0) &= x_0
\end{aligned}
\]

The solution to this matrix differential equation is similar to the scalar case.
The solution is given by $x(t) = e^{At}x_0$. Bashing out the calculations gives:
\[
\begin{aligned}
    x(t) &= e^{At}x_0 \\
    &= \frac{1}{2}
    \begin{bmatrix}
        e^t + e^{3t} & e^t - e^{3t} \\
        e^t - e^{3t} & e^t + e^{3t}
    \end{bmatrix}
    \begin{bmatrix}
        -1 \\
        1
    \end{bmatrix} \\
    &=
    \begin{bmatrix}
        -e^{3t} \\
        e^{3t}
    \end{bmatrix}
\end{aligned}
\]

The zero-input response is given by:
\[
\begin{aligned}
    y(t) &= Cx(t) \\
    &=
    \begin{bmatrix}
        0 & 3
    \end{bmatrix}
    \begin{bmatrix}
        -e^{3t} \\
        e^{3t}
    \end{bmatrix} \\
    &= 3e^{3t}
\end{aligned}
\]

The zero-input trajectory is therefore
\begin{empheq}[box=\fbox]{align*}
    x_{z-i}(t) &=
    \begin{bmatrix}
        -e^{3t} \\
        e^{3t}
    \end{bmatrix} \\
    y_{z-i}(t) &= 3e^{3t}
\end{empheq}

\subsection{}
The zero-state trajectory problem is given by:
\[
\begin{aligned}
    \dot{x} &= Ax + Bu \\
    x(0) &= 0   
\end{aligned}
\]

The solution to this matrix differential equation is similar to the scalar case where an 
integrating factor approach is used. The solution is given by $x(t) = \int_{0}^{t} e^{A(t-\tau)}Bu(\tau) d\tau$.

First step to evaluating this \textit{garbage} is to evaluate the integrand:
\[
\begin{aligned}
    e^{A(t-\tau)}Bu(\tau) &= 
    \frac{1}{2}
    \begin{bmatrix}
        e^{t-\tau} + e^{3(t-\tau)} & e^{t-\tau} - e^{3(t-\tau)} \\
        e^{t-\tau} - e^{3(t-\tau)} & e^{t-\tau} + e^{3(t-\tau)}
    \end{bmatrix}
    \begin{bmatrix}
        2 \\
        0
    \end{bmatrix}
    u(\tau) \\
\end{aligned}
\]

Conveniently, $u(\tau) = 1_{+}(\tau)$, so the integrand becomes:
\[
\begin{aligned}
    e^{A(t-\tau)}Bu(\tau) &= 
    \frac{1}{2}
    \begin{bmatrix}
        e^{t-\tau} + e^{3(t-\tau)} & e^{t-\tau} - e^{3(t-\tau)} \\
        e^{t-\tau} - e^{3(t-\tau)} & e^{t-\tau} + e^{3(t-\tau)}
    \end{bmatrix}
    \begin{bmatrix}
        2 \\
        0
    \end{bmatrix} \\
    &=
    \begin{bmatrix}
        e^{t-\tau} + e^{3(t-\tau)} \\
        e^{t-\tau} - e^{3(t-\tau)}
    \end{bmatrix}
\end{aligned}
\]

Taking the integral term-by-term gives:
\[
    \arraystretch{1.5}
    \begin{aligned}
    \int_{0}^{t} e^{A(t-\tau)}Bu(\tau) d\tau &=
    \begin{bmatrix}
        \int_{0}^{t} e^{t-\tau} + e^{3(t-\tau)} d\tau \\
        \int_{0}^{t} e^{t-\tau} - e^{3(t-\tau)} d\tau
    \end{bmatrix} \\
    &=
    \begin{bmatrix}
        \int_{0}^{t} e^{t-\tau} d\tau + \int_{0}^{t} e^{3(t-\tau)} d\tau \\
        \int_{0}^{t} e^{t-\tau} d\tau - \int_{0}^{t} e^{3(t-\tau)} d\tau
    \end{bmatrix} \\
    &=
    -
    \begin{bmatrix}
        e^{t-\tau} \big|_{0}^{t} + \frac{1}{3}e^{3(t-\tau)} \big|_{0}^{t} \\
        e^{t-\tau} \big|_{0}^{t} - \frac{1}{3}e^{3(t-\tau)} \big|_{0}^{t}
    \end{bmatrix} \\
    &=
    -
    \begin{bmatrix}
        (1 - e^{t}) + \left(\frac{1}{3} - \frac{1}{3}e^{3t}\right) \\
        (1 - e^{t}) - \left(\frac{1}{3} - \frac{1}{3}e^{3t}\right)
    \end{bmatrix} \\
    &=
    \begin{bmatrix}
        e^{t} + \frac{e^{3t}}{3} - \frac{4}{3} \\
        e^{t} - \frac{e^{3t}}{3} - \frac{2}{3}
    \end{bmatrix}
\end{aligned}
\]

The zero-state response is given by $y_{z-s} = Cx_{z-s} + Du$. Since $D = 0$, the output is simply:
\[
\begin{aligned}
    y_{z-s}(t) &= Cx_{z-s}(t) \\
    &=
    \begin{bmatrix}
        0 & 3
    \end{bmatrix}
    \begin{bmatrix}
        e^{t} + \frac{e^{3t}}{3} - \frac{4}{3} \\
        e^{t} - \frac{e^{3t}}{3} - \frac{2}{3}
    \end{bmatrix} \\
    &= 3e^{t} + e^{3t} - 2
\end{aligned}
\]

The zero-state trajectory is therefore
\begin{empheq}[box=\fbox]{align*}
    x_{z-s}(t) &=
    \begin{bmatrix}
        e^{t} + \frac{e^{3t}}{3} - \frac{4}{3} \\
        e^{t} - \frac{e^{3t}}{3} - \frac{2}{3}
    \end{bmatrix} \\
    y_{z-s}(t) &= 3e^{t} + e^{3t} - 2
\end{empheq}

\subsection{}
Lastly, summing the zero-input and zero-state trajectories gives the full trajectory:
\begin{empheq}[box=\fbox]{align*}
    x(t) &= x_{z-i}(t) + x_{z-s}(t) \\
    &=
    \begin{bmatrix}
        -e^{3t} \\
        e^{3t}
    \end{bmatrix}
    +
    \begin{bmatrix}
        e^{t} + \frac{e^{3t}}{3} - \frac{4}{3} \\
        e^{t} - \frac{e^{3t}}{3} - \frac{2}{3}
    \end{bmatrix} \\
    &=
    \begin{bmatrix}
        e^{t} - \frac{2e^{3t}}{3} - \frac{4}{3} \\
        e^{t} + \frac{2e^{3t}}{3} - \frac{2}{3}
    \end{bmatrix}
\end{empheq}

The full response is given by $y(t) = Cx(t) + Du(t)$. Since $D = 0$, the output is simply:
\begin{empheq}[box=\fbox]{align*}
    y(t) &= Cx(t) \\
    &=
    \begin{bmatrix}
        0 & 3
    \end{bmatrix}
    \begin{bmatrix}
        e^{t} - \frac{2e^{3t}}{3} - \frac{4}{3} \\
        e^{t} + \frac{2e^{3t}}{3} - \frac{2}{3}
    \end{bmatrix} \\
    &= 3e^{t} + 2e^{3t} - 2
\end{empheq}

\subsection{}
The code is given by:
\lstinputlisting[language=Matlab]{Questions/Code/a3q1.m}