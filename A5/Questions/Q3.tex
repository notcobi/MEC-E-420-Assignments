\section{}
% Consider the transfer function
% G(s) = s + 1
% s
% 2 + 2s + 3
% .

Consider the transfer function
\[
    G(s) = \frac{s + 1}{s^2 + 2s + 3}
\]

\subsection{}
By hand, obtain a state-space realization of this transfer function.

\textbf{Solution} \\
From $Y(s) = G(s) U(s)$,
\begin{align*}
    Y(s) = (s+1) \underbrace{\frac{1}{s^2 + 2s + 3} U(s)}_{:= V(s)}
\end{align*}

Define an intermediate signal V(s). This leads to:
\begin{align*}
    \ddot{v} + 2 \dot{v} + 3 v &= u \\
    Y(s) &= (s+1) V(s) \implies y = \dot{v} + v \\
\end{align*}

Let $x_1 = v$ and $x_2 = \dot{v}$. Then,
\begin{align*}
    \dot{x_1} &= x_2 \\
    \dot{x_2} &= -3 x_1 - 2 x_2 + u \\
    y &= x_2 + x_1
\end{align*}

In state-space form,
\begin{empheq}[box=\fbox]{align*}
    \dot{x} &=
    \underbrace{\begin{bmatrix}
        0 & 1 \\
        -3 & -2 \\
    \end{bmatrix}}_{A}
    \begin{bmatrix}
        x_1 \\
        x_2 \\
    \end{bmatrix}
    + 
    \underbrace{\begin{bmatrix}
        0 \\
        1 \\
    \end{bmatrix}}_{B}
    u \\
    y &=
    \underbrace{\begin{bmatrix}
        1 & 1 \\
    \end{bmatrix}}_{C}
    \begin{bmatrix}
        x_1 \\
        x_2 \\
    \end{bmatrix}
    +
    \underbrace{0}_{D}
\end{empheq}

\subsection{}
Use MATLAB to obtain a (different) realization of the system

\textbf{Solution} \\
From Matlab,
\begin{verbatim}
A =

    -2    -3
     1     0


B =

     1
     0


C =

     1     1


D =

     0
\end{verbatim}

The code used to generate this is:
\lstinputlisting[language=Matlab]{Questions/Code/a5q3b.m}

\subsection{}
Demonstrate that the realizations in (a) and (b) both correspond to the same $G(s)$. Note you
can employ MATLAB's symbolic toolbox for this.

\textbf{Solution} \\
Recall the definition of the transfer function:
\begin{align*}
    G(s) &= C(sI - A)^{-1} B + D
\end{align*}

Define the hand solution as $G_1(s)$ and the Matlab solution as $G_2(s)$. Then from Matlab,
\begin{verbatim}
G1 =

s/(s^2 + 2*s + 3) + 1/(s^2 + 2*s + 3)
    
    
G2 =
    
s/(s^2 + 2*s + 3) + 1/(s^2 + 2*s + 3)
\end{verbatim}
\[\boxed{\text{It is clear that } G_1(s) = G_2(s)}\]
The code used to generate this is:
\lstinputlisting[language=Matlab]{Questions/Code/a5q3c.m}


