\section{}
Using the Laplace Transform approach, compute the (total) response $y(t)$ of the state-space system
\[
\begin{aligned}
    \begin{bmatrix}
        \dot{x_1} \\
        \dot{x_2}
    \end{bmatrix}
    &=
    \begin{bmatrix}
        2 & -1 \\
        -1 & 2
    \end{bmatrix}
    \begin{bmatrix}
        x_1 \\
        x_2
    \end{bmatrix}
    + 
    \begin{bmatrix}
        2 \\
        0
    \end{bmatrix}
    u,\;\;\;\;\;\;
    \begin{bmatrix}
        x_1(0) \\
        x_2(0)
    \end{bmatrix}
    =
    \begin{bmatrix}
        -1 \\
        1
    \end{bmatrix},\;\;\;\;\;\;\;\;
    u(t) = 1_{+}(t) \\
    y &=
    \begin{bmatrix}
        0 & 3
    \end{bmatrix}
    \begin{bmatrix}
        x_1 \\
        x_2
    \end{bmatrix}
\end{aligned}
\]
Note the solution should match Assignment \#3.

\textbf{Solution}\\
Observe that the system is already in state-space form. We can immediately utlize the Laplace Transform response equation:
\[
\begin{aligned}
    Y(s) &= C(sI - A)^{-1}x_0 + C(sI - A)^{-1}B U(s) + D U(s) \\
\end{aligned}
\]

Computing the Laplace Transform of the input $u(t)$,
\[
\begin{aligned}
    U(s) &= \mathcal{L}\left\{ 1_{+}(t) \right\} \\
    &= \frac{1}{s}
\end{aligned}
\]

By Matlab,
\begin{verbatim}
Y =
 
2*exp(3*t) + 3*exp(t) - 2
\end{verbatim}
Written nicely,
\begin{empheq}[box=\fbox]{align*}
y(t) &= 2e^{3t} + 3e^t - 2
\end{empheq}

The code used to generate the above is
\lstinputlisting[language=Matlab]{Questions/Code/a4q2.m}